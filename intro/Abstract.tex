\begin{resume}
Ce rapport présente mon projet de fin d’étude axé sur l’automatisation des tests E2E et des tests de non-régression dans le contexte d’un projet de monitoring. Il met en évidence l’importance cruciale des tests à toutes les étapes du cycle de développement, en particulier les tests de non-régression, indispensables pour maintenir l’indépendance des différents modules de l’application. L’automatisation de ces tests est présentée comme une approche visant à réduire la charge de travail et à améliorer la détection des anomalies. Parallèlement, les tests E2E automatisés sont déployés pour vérifier le bon fonctionnement global de l’application et garantir la conformité aux exigences du cahier des charges ainsi qu’aux SLA. En outre, ce projet vise à favoriser une meilleure application des principes Agile dans le processus de développement, contribuant ainsi à une approche plus efficace et réactive dans le cadre du monitoring des systèmes.
\end{resume}